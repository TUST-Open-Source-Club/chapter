\documentclass{gbt9704}
\title{天津科技大学开放原子开源协会(筹备)章程草案}

\begin{document}
\maketitle


第一条~ 本社团全称为“天津科技大学开放原子开源协会”,是隶属于天津科技大学人工智能学院的学术科技类学生社团组织,需在遵守学校有关规定的前提下活动。


第二条~ 本社团的宗旨是推动开源技术普及与实践,宣传开源文化,培养开源社区贡献能力,促进学术创新与技术应用融合发展。


第三条~ 天津科技大学在校生认可本章程并提出申请,经按照活跃会员大会规定的程序审核通过,即成为本社会员。


第四条~ 会员享有下列权利:
\begin{enumerate}
    \item 参与社团组织的各类活动;
    \item 对社团工作提出建议与批评;
    \item 符合活跃会员标准时行使表决权、选举权及被选举权。
\end{enumerate}


第五条~ 会员应履行下列义务:
\begin{enumerate}
    \item 遵守章程及社团管理制度;
    \item 参与社团活动并行使投票权;
    \item 维护社团声誉与合法权益。
\end{enumerate}


第六条~ 一学期内参与社团正式活动次数占比大于等于25\%者,或者
参与为期长于一个月的长期活动一次以上者,为活跃会员。只有活跃
会员可以参加活跃会员大会并行使表决权,非活跃会员可以自愿列席。
新加入未满一学期的会员,可以列席活跃会员大会。


本社团的创始会员,自动成为活跃会员,且不受前款所述对于新加入会员的限制;但在本社团第一次注册前参加活动少于二次者除外。


活跃会员在下一学期参与活动比例低于25\%且未参与为期长于一个月的长期活动一次以上的,其活跃会员资格自然丧失。


活跃会员名单由主席团公告。


第七条~ 活跃会员大会是社团的决策机关,行使下列职权:
\begin{enumerate}
    \item 决定社团发展规划及重大事项;
    \item 审查和批准社团预算(含开放原子基金会支持积分);
    \item 经三分之二以上活跃会员出席,并经出席的活跃会员的三分之二以上多数赞成:
        \begin{enumerate}
            \item 修改本章程
            \item 开除会员
        \end{enumerate}
    \item 选举主席团;
    \item 根据主席团的提名,决定财务专员的人选。
\end{enumerate}


决议可通过线上投票方式作出。

活跃会员大会举行选举时,
有投票权的会员有权了解候选人情况、要求改变候选人、
不选任何一个候选人或者另选他人。活跃会员大会举行其他表决时,
有投票权的会员有权了解草案内容和相关情况,有权要求修改草案,有权
投赞成、反对和弃权票。除本章程另有规定外,活跃会员大会举行
选举和表决须有半数以上有投票权的会员参与,并以出席人员的半数以上
赞成票作出决定。


活跃会员大会可以随时罢免任何主席团成员和财务专员。罢免主
席团成员职务,必须经过学校规定的程序进行。


第八条~ 主席团是社团的执行机关,由活跃会员大会选举产生,
任期一年。


第九条~ 主席团履行下列职责:
\begin{enumerate}
    \item 召集并主持活跃社员大会,预先讨论拟提请活跃社员大会讨论的问题,决定是否列入议程;
    \item 执行活跃会员大会决议;
    \item 策划和组织日常社团活动;
    \item 提名财务专员人选;
    \item 活跃社员大会决定由主席团行使的其他职权,以及
    为完成学校有关部门交办的任务所需要的职权。
\end{enumerate}


第十条~ 主席团主席由主席团内部选举产生,召集并主持会议;
主席的主席团成员职务被罢免时,其主席职务自动终止。

主席团成员任满一届以上后,继续担任主席团成员应当根据学
校有关规定通过审批。主席团成员未任满一届的,视为任满一
届任期。

第十一条~ 按照学校有关规定,本社团设立财务专员,行使下
列职权:
\begin{enumerate}
    \item 保管社团全部财产;
    \item 按社团预算和主席团决议支配资金和开放原子基金
    会资助;
    \item 每季度公示资产使用明细(含基金会资助);
\end{enumerate}


财务专员不是主席团成员时,列席主席团会议。


第十二条~ 社团经费来源包括:
\begin{enumerate}
    \item 学校及学院专项资助;
    \item 开放原子基金会等社会支持;
    \item 其他合法和合乎学校规定的收入。
\end{enumerate}


第十三条~ 本章程修订须经活跃会员大会三分之二以上出席,出
席活跃会员三分之二以上多数赞成表决通过。


第十四条~ 本章程自本社团第一次注册通过之日起施行。

第十五条~ 天津科技大学及其有权内设机构对本社团的管理行为,
不受本章程约束,相关事宜由学校与社团讨论确定。
\end{document}